\section*{謝辞}
\begin{center}
この本を読んでくださる方に \\
気力をくれる友人に \\
大切な人に \\
感謝と本書をささげます
\end{center}

\section*{前書き}

本書は、SQLでわかりにくい概念をゆるく説明するシリーズの第2弾です。
今回は、自己結合についてと、UPDATEの概念、アトミックとトランザクション、という三つのテーマをとりあげています。

SELECTは、テーブルとして適切なモデリングができれば、いろいろな問題を解くことにつかうことができます。いささか技芸的な面はありますが、SELECTの考え方を理解する一助になれば幸いです。

また、SQLは関数である、という主張の元で、UPDATEはどのように解釈すればよいか、という視点で、UPDATEの説明を試みました。
レコード単体の書き換えとして見られるUPDATEですが、テーブルの状態繊維という視点で見ることで、理解が進めばと思います。

\section*{本書の内容}

\paragraph{第一章}

自己結合と、その利用について説明します。同じ構造、同じレコードのテーブルと結合するということはどういういみがあるのか、いう説明を行い、内部結合、外部結合それぞれを利用して問題を解く例を示します。

\paragraph{第二章}

UPDATEの、テーブル全体への作用がどのような性質を持つかについて、説明します。
次に、UPDATEなど、テーブル全体への作用を担保する仕組みと概念として、アトミックとトランザクションについて説明します。

\section*{免責事項}
本書に書いてあることは、筆者知識のレベルでまとめたものです。ですが、内容が正しいとは言い切れません。初版でも改訂版でも相当やらかしています。また、学校のレポート、業務などのコードを書く際に、本書の内容を信じて書いて損害が生じても、筆者にその責任はありません。

くれぐれも、自己責任と十分な検証の上、ご利用ください。

\section*{表紙イラスト}
ゆうちゃん (コース英知)