\chapter{あとがき}

\section*{メイ・カートミル(仮名)とマージ・ニコルス(仮名)の秋葉原での会話}

\begin{quotation}
\noindent
{\bf メイ}「マージさん、今回このコーナーお休みなんですか?」 \\
{\bf マージ}「残り時間の都合でね」 \\
\end{quotation}


\section*{後書き}

お兄ちゃん、さすがにSQLでハノイの塔の解法は書けなかったよ……

枕詞のあとですが、まずは、ゆうちゃんさん、表紙をありがとうございます。
まさかあの頃のときめきがよみがえる表紙になるとは思いませんでした。余談ながら、わたしがPCエンジン版で初めてプレイしたとき、館林さんを出しました。

というわけで、インフラエンジニアの毒舌な妹(@infra\_imouto)です.

今回は、一発芸にはしったような気もします。ぶっちゃければ、三平方の定理とか、n-Queenとか、Haskellで解く方がかしこいです。SQLは遅延評価とかないからなぁ。

トランザクションは、ノイマン型で実行するからいるんだ、という論法になりましたがm、みんな、コンピュータの動作をどこまでイメージしているのかな、と思ったり。

でも、UPDATEって空間内での要素を移動させること、って考えたらわかりやすい、かな?
そういういろんなことを、薄いながらつめこんだ本になった、はずです。

お兄ちゃん、ごめん、今までで一番薄い本になった

\begin{flushright}
2019年9月22日 \\
インフラエンジニアの毒な妹 \\
\end{flushright}

いつもどおりですが、まずは表紙のゆうちゃんさんへの謝辞から。
まさかあの頃のときめきが……これはインフラ妹が書いていますね。

今回も、共著者として、、主に図版と検証を担当した、ありすゆうです。
4-QueenとかSELECTでこれやるのか、と思っていたら、本当に解けるんだ、と、SELECTの奥深さをあらためて感じたりする作業でした。

でも、書いてあるとおり、SQLはモデリングを決め打たなければならないので、柔軟な処理ならHaskellなりRなりを使うべき、というのもインフラ妹が書いてある通りです。

SQLは数学的、集合論的なイメージが必要な言語、というのは著者一同一致するところです。
そのイメージを少しでも伝えられれば、著者一同それに勝る喜びはありません・

\begin{flushright}
2019年9月22日 \\
ありす ゆう
\end{flushright}


%\newpage
% ここまでで160ページ鳴ったのでブランクなし
% 1ページブランクを入れる

\thispagestyle{empty}
\mbox{}
\newpage
\clearpage

% PICOはノンブルがいる
%\thispagestyle{empty}
\mbox{}
\newpage
\clearpage


% PICOはノンブルがいる
%\thispagestyle{empty}

\vspace*{\fill}
\begin{tabular}{ll} \toprule
筆者 & インフラエンジニアの毒舌な妹 ありす ゆう\\
発行 & AliceSystem \\
連絡先 & aliceyou@alicesystem.net \\
URL & http://aliceyou.air-nifty.com/onesan/ \\
初版発行日 & 2019年9月22日 \\
印刷所 & 日光企画  \\ \bottomrule
\end{tabular}