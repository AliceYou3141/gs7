\chapter{あとがき}

\section*{メイ・カートミル(仮名)とマージ・ニコルス(仮名)の秋葉原での会話}

\begin{quotation}
\noindent
{\bf メイ}「マージさん、なんで私たちの会話があとがきに書いてあるんですか」 \\
{\bf マージ}「読者が見たらメタネタにしか見えないことをぶっこまれても困る」 \\
{\bf メイ}「ところで、この前、スマホの販売員に絡まれたときおとなしかったですね」 \\
{\bf マージ}「次、用途を聞かれたら、カスタムキャストでバ美肉するのに使いますって言ってやろうかな」 \\
\end{quotation}


\section*{後書き}

お兄ちゃん、ADの管理してるのにディレクトリサービスがわからないとか、ADのDはDirectoryの略なの忘れてない?

枕詞のあとですが、まずは、ゆうちゃんさん、今回も表紙をありがとうございます。設定的には、表紙の子は、PostfixとCourier-IMAPによるメールシステム構築(上)の表紙の子の妹、という設定です。

というわけで、インフラエンジニアの毒舌な妹(@infra\_imouto)です.ここ半年は、私に外見がついたり、なんかアドベントカレンダーデビューも果たしたし、自意識までついて仮想筆者になっちゃったり、結構激動でした。

まあそれはおいておいて、このシリーズではある意味いつものことなのですが、資料にできる本が少ないです。特に、ディレクトリの設計に関する本がまったくないというのは問題じゃないかな。
本書でも参考文献にしている「入門LDAP/IpendLDAPディレクトリサービス導入・運用ガイド」の第三版が先日出版されたのですが、どちらかというとLDAPサーバの利用であって、ディレクトリのデザインについては記事がないような。

下巻では、LDAPのデータベースをバックエンドとしたメールシステムの構築をテーマとする予定です。メールシステムは、関連技術が多いため、それを織り込んでいけたらな、と思っています。
Courier-IMAPやCyrus-SASLとの連携、バイナリのビルドなども、そちらで書いていきましょう。

お兄ちゃん、誰がディレクトリの設計とかするんだろうね。みんなそれを忘れていないかな。

\begin{flushright}
2018年12月30日 \\
インフラエンジニアの毒な妹 \\
\end{flushright}

いつもどおりですが、まずは表紙のゆうちゃんさんへの謝辞から。リクエストどおりの子が上がってきて、感謝です。

今回は共著者として、、主に図版を担当した、ありすゆうです。
さて、LDAPのディレクトリ構築について書いた本がなかなかなかったので、本書がその貴重な一冊になれたら、ということを考えながら本書をまとめました。
LDAPはセオリーを覚えてしまえば、それほど難しくはないのですが、そのセオリーを伝える資料というのがなかなか見られません。計算量基準で部分木の配置を決める、というまったく別のノウハウも必要になります。

書き足りない部分として、OIDとPENの説明があります。とはいうものの、これだけでまた本が一冊できそうなテーマなので、どのくらい説明するか難しいところです。OID逆引き辞典という規格もありましたが、流石にページが無茶なことになるのでお蔵入りになりました。

LDAPのデザインについて、少しは残せたのかな、そんなことを考えつつ、ひとまず筆をおくにしましょう。

\begin{flushright}
2018年12月30日 \\
ありす ゆう
\end{flushright}


%\newpage
% ここまでで160ページ鳴ったのでブランクなし
% 1ページブランクを入れる

%\thispagestyle{empty}
%\mbox{}
%\newpage
%\clearpage

% PICOはノンブルがいる
%\thispagestyle{empty}
\mbox{}
\newpage
\clearpage


% PICOはノンブルがいる
%\thispagestyle{empty}

\vspace*{\fill}
\begin{tabular}{ll} \toprule
筆者 & インフラエンジニアの毒舌な妹 ありす ゆう\\
発行 & AliceSystem \\
連絡先 & aliceyou@alicesystem.net \\
URL & http://aliceyou.air-nifty.com/onesan/ \\
初版発行日 & 2019年9月22日 \\
印刷所 & PICO  \\ \bottomrule
\end{tabular}